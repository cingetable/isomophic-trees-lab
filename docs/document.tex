\documentclass{article}
\usepackage[utf8]{inputenc}
\usepackage[T2A]{fontenc}    
\usepackage[utf8]{inputenc}  
\usepackage[english,russian]{babel}
\usepackage{indentfirst}
\usepackage{tikz}  
\usetikzlibrary{graphs}
\title{Изоморфизм деревьев}
\author{Баздуков Валентин}
\date{Владивосток, 2022}


\begin{document}
	
	\maketitle
	\newpage
	\tableofcontents
	\newpage
	\section{Введение}
	
	В данной работе будет рассматриваться проблема изоморфизма деревьев. Основные идеи были взяты из книги Ахо А., Хопкрофта Дж., Ульмана Дж. "Построение и анализ вычислительных алгоритмов".
	
	На практике алгоритм хорошо зарекомендовал себе в молекулярной химии, где зачастую изучаемые структуры представляют собой деревья с миллионами вершин. Таким образом задача проверки изоморфности двух структур сводится к проверке изоморфности двух деревьев.
	\section{Теория}
	\subsection{Изоморфзим графов}
	Так как корневое дерево является частным случаем графа, то сначала стоит дать определение изоморфизма графов.
	\newline
	\newline
	\textbf{Определение: \textit {Изоморфизмом графов}} 
	\textit{$G_{1}(V_{1},E_{1})$ и $G_{2}(V_{2},E_{2})$ называется биекция между наборами вершин $\varphi$: $V_{1} \rightarrow V_{2}$ такая что: }
	\textit{$$\forall u,v \in V_{1} \quad (u,v) \in E_{1} \Leftrightarrow (\varphi(u), \varphi(v)) \in E_{2}$$}
	\newline
	Выделяют несколько фактов об изоморфизме графов:
	\begin{itemize}
		\item До сих пор неизвестно, является ли алгоритм изоморфизма графов NP -- полной задачей
		\item Существует алгоритм с полиноминальным временем для различных подклассов графов, таких как деревья
	\end{itemize}
	\subsection{Корневые деревья}
	\textbf{Определение: \textit {Корневое дерево (V,E,\textit{r})}} 
	\textit{это дерево (V,E), в котором определен корень r $\in$ V.}
	
	
	\textbf{Определение: \textit {Изоморфизмом корневых деревьев}} 
	\textit{$T_{1}(V_{1},E_{1},r_{1})$ и $T_{2}(V_{2},E_{2},r_{2})$ называется биекция между наборами вершин $\varphi: V_{1} \leftarrow V_{2}$} такая что:
	\textit{$$\forall(u,v)\in V_{1} \quad (u,v)\in E_{1} \Leftrightarrow (\varphi(u),\varphi(v)) \in E_{2}\quad \fbox{$\varphi(r_{1} = r_{2})$} $$}
	\newline
	Так как корневые деревья дают нам гораздо больше информации о самих себя, нежели графы, то существует алгоритм, работающий за полиноминальное время.
	\section{Алгоритм AHU}
	Данный алгоритм был разработан Ахо А. Хопкрофтом Дж., Ульман Дж. для определения изоморфности двух деревьев. 
	
	У этого алгоритма есть 2 основных свойства:
	\begin{itemize}
		\item Определяет изоморфизм корневого дерева за O(|V|)
		\item Использует полную историю потомков степеней вершин в своей работе
	\end{itemize}
	
	\textbf{Основная идея алгоритма} состоит в том, чтобы связывать с каждой вершиной кортеж, который хранит полную историю его потомков
	\subsection{Реализация}
	\subsubsection{Поиск центров дерева}
	По определению изоморфизма 
	\subsubsection{Хеширование узлов}
	\subsubsection{Оптимизация}
	\section{Тестирование}
	\subsection{О способе представления дерева}
	\subsection{Простые тесты}
	\subsection{Тесты на обход наивной реализации}
	\section{Заключение}
	\section{Список источников}
\end{document}
